\chapter{Wyniki i wnioski}

Aby sprawdzić działanie i dokładność systemu została zaimplementowana możliwość zapisu użytego obliczonego wektora cech na karcie SD. Następnie został obliczony przykładowy błąd względny między wektorem cech wyliczonym w implementacji programowej a uzyskanym z sytemu wizyjnego. Błąd oscyluje w granicy \(10^{-6}\) co czyni go marginalnym i najprawdopodobniej wynika z różnić użytych bibliotek numerycznych.

<TU WSTAW WYKRES>

Na przebadanie jednego okna zaproponowany system procesorowy potrzebuje 75ms  (dla porównania te same obliczenia w pakiecie Matlab zajmują około 23 ms). Dzięki zastosowaniu sprzętowego wyszukiwania ROI zadanie sytemu procesorowego zostało ograniczone do obliczenia jednego okna z największym prawdopodobieństwem zawierania w sobie przechodnia. Kamera termowizyjna będąca źródłem sygnału dla wzorca probabilistycznego pracuję z prędkością 9 klatek na sekundę dając w przybliżeniu 111 ms na zbadanie danego okna więc system procesorowy mieści się w tych ramach czasowych z dużym zapasem. 

\begin{table}[]
\centering
\caption{Wykorzystane zasoby logiki programowalnej.}
\label{tab:resources}
\begin{tabular}{|l|l|l|l|}
\hline
Resource & Utilization & Available & Utilization \% \\ \hline
LUT & 12583 & 17600 & 71,49 \\ \hline 
LUTRAM & 617 & 6000 & 10,28 \\ \hline 
FF & 19924 & 35200 & 56,60 \\ \hline
BRAM & 25,50 & 60 & 42,50 \\ \hline
DSP & 36 & 80 & 45,00 \\ \hline
IO & 43 & 100 & 43,00 \\ \hline
BUFG & 7 & 32 & 21,88 \\ \hline
MMCM & 1 & 2 & 50,00 \\ \hline
PLL & 1 & 2 & 50,00 \\ \hline
\end{tabular}
\end{table}